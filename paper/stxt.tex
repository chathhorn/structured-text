% \begin{abstract}
% 
% \end{abstract}

\section{Intro}

Programmable logic controllers (PLCs) are used for industrial automation in many
safety- and security-critical contexts. The IEC 61131-3 standard informally
describes five languages intended for programming PLCs. Three of these are
graphical---ladder diagrams (LD), function block diagrams (FBD), and sequential
function charts (SFC)---while two are textual---structured text (ST) and
instruction lists (IL). And often, PLCs are programmed using a combination of
these languages. As the scale and complexity of industrial automation grows, so
do the scale and complexity of these programs. When these programs are deployed
in a critical role, we would like mathematically-rigorous confidence in their
correctness.

We will explore techniques for high-assurance PLC programming in the IEC 61131-3
languages: formal verification of PLC programs or an IEC 61131-3 language
implementation, model checking, abstract interpretation, or other static or
dynamic analyses of IEC 61131-3 programs.

\section{Motivation}

\section{Structured Text}

\section{LTL}

\section{Previous work on high-assurance PLC programming}

\subsection{High-assurance programming}

In general, the goal of high-assurance programming techniques is to increase our
confidence in the correctness of programs. Given that programs are physical
manifestations of human passions existing in this fallen world, their rightness
and trueness will always, at best, be questionable. But using careful reasoning
and various mathematical tools, it is still possible to gain a great deal of
confidence that a program will never betray our intentions for it.

Through the ages, many such techniques have been developed. One can start simply
by interrogating the program: if we give it $x$, will it return $y$? What
about if we give it $z$? But for most progams, the possible state space of
inputs is so large that testing even a large fraction of them is infeasible. And
even if every input can be tested, changing the environment the program runs in
might also affect its behavior. So optimally we would like to go beyond simple
testing to directly reason about the program itself.

But first: what does it mean for a program to be correct? Even that can be a
hard question to answer (and bugs can exist in our definition of
correctness---or our understanding of requirements---just as easily as in the
program itself).

Anyhow, some techiques:

\begin{itemize}

\item Specification. Spend the time to actually think about what the program
should do and write it out in some formal (e.g., logic, math, type system) or
informal way.

\item Redundancy. Implement the program two or more times and compare the
implementations. Or implement the program once in a programming language and
another time in a logic or modeling language.

\item Testing.

\item Model checking.

\item Formal verification.

\item Other static or dynamic analysis: control-flow analysis, abstract
interpretation, type checking, lint checks, runtime verification, monitoring.

\end{itemize}

\subsection{Formal semantics for PLC programming languages}

In order to directly reason about PLC programs, we need to know what they mean
in some rigorous way. Programming language semantics typicaly fall into three
camps: denotational, operational, or axiomatic~\cite{hoare:axiomatic}. Each of
these approaches can be understood as a translation from statements in the
programming language to its ``meaning'' in terms of some mathematical or logical
structure. The approaches differ in what that ``meaning'' actually is: algebraic
objects or domains (denotational semantics), operations on an abstract machine
(operational), or assertions about how the program state evolves (axiomatic).

Previous work on the semantics of PLC languages includes:

\begin{itemize}

\item Coq semantics: Biha and Blech~\cite{biha:plc_sem1, biha:plc_sem2,
biha:plc_sem3}. They give a formal semantics for IL and sequential function
charts (SFC) and a tool for automatically generating SFC representations from a
graphical description. Also they prove safety properties of a PLC in a real
industrial demonstrator.

% Note: should look into this. The ``industrial demonstrator'' might be useful.
% The Coq semantics might be interesting. I'm curious how they formalized SFC as
% well as what the hell this tool is.

\item Petri net semantics: Heiner and Menzel~\cite{heiner:petri}. They give a
semantics for the IL language.

% Note: Petri nets, ugh.

\item $\mathbb{K}$ framework operational semantics: Huang et al.~\cite{huang:kst}.

% Note: they cite me. :)

\end{itemize}

\subsection{Symbolic execution, testing}

\begin{itemize}

\item Concolic test generation: Bohlender et al.~\cite{bohlender:concolic}.

\end{itemize}

\subsection{Monitoring}

Also runtime verification.

\begin{itemize}

\item IDS: Had\u ziosmanovi\'c and Sommer. They appear to take advantage of the
generally highly-regular and predictable nature of many industrial control
systems while monitoring for anomalies. They apparently test their approach on a
water treatment plant.

% Note: maybe a broader scope, but it might be fruitful to think of the whole
% industrial process in which the PLC resides as the system to be verified.

\end{itemize}

